\documentclass{tufte-handout}
\usepackage{amsmath,amsthm}

%%% This file is generated by Makefile.
%%% Do not edit this file!
%%%
	\gdef\GITAbrHash{d819d05}	\gdef\GITAuthorDate{2017-09-29}	\gdef\GITAuthorName{Arcombe}  % For version control

\usepackage{booktabs}
\usepackage{graphicx}
\usepackage{tikz}

\newtheorem{claim}{Claim}[section]
\title{Treewidth}
\date{\GITAuthorDate, rev. \GITAbrHash}
\author{}

\begin{document}
\maketitle

Implement an algorithm for independent set using dynamic programming over a (given) tree-decomposition.

\emph{2017 is the second time we try this exercise. Not all problems from last year are resolved.}

\subsection{The algorithm}

The algorithm takes as input an unweighted, undirected graph $G$ and a tree-decomposition $T$ of $G$ with width $w$.
A detailed explanation of the algorithm can be found in \emph{Tree Decompositions of Graphs}, Section 10.4 of Kleinberg and Tardos, \emph{Algorithms Design}, Addison--Wesley 2005.

The input files come in pairs with extension {\tt .td} and {\tt .gr}, respectively.
The format is described in {\tt data/README.md}.


You need to achieve a running time of $\exp(O(w))\operatorname{poly}(n)$; a straightforward implementation of the pseudo-code in the book will achieve that, as analysed at the end of Section 10.4.

\paragraph{My thoughts about implementation.}
I parse both $G$ and $T$ as graphs.
Note that the input representation of $T$ is just an undirected, connected graph without cycles, so we can pick any node as the root $r$ of the tree decomposition.
From $r$, I perform a (very simple) graph traversal that allows me to associate with each node $t$ of $T$ the list of its children (in $T$) and a topological ordering.
I ended up associating the following information with each node $t$ in the tree-decomposition:
\begin{enumerate}
\item A list of its children.
\item The piece $V_t$ (sometimes called `bag' in the literature), as a set of vertices from $G$.
\item A table of $2^{w+1}$ values $f_t(U)$, for each $U\subseteq V_t$. 
  Initially, these values are undefined.
  They get filled in by the dynamic programming algorithm.
\end{enumerate}

The graphs called {\tt web$k$} ($k\in\{1,\ldots,4\}$) and {\tt eppstein} are meant to be useful for initial debugging.

A lot of my attention was spent on handling sets.
(We need to iterate over subsets, take set intersections, and test for set equality.)
I can see two approaches for this.
\begin{enumerate}
  \item At node $t$, rename the vertex names so as to identify $V_t$ with $\{0,\ldots, w\}$ and store each subset $U\subseteq \{0,\ldots,w\}$ as a bit string $b_0\cdots b_w$ where $b_i=1$ if and only if $i\in U$.
    If you choose this implementation, you are allowed to assume that $w$ is never larger than the word length on your machine.
    Thus, such a representation can be stored in a single machine word.
    The set operations now become (hairy but compact) bit fiddling operations.
    This solution is very fast, and a low level language like C works extremely well for it.
    Table lookup is just array access, and iteration over subsets is (careful) incrementation.
    The difficulty here is to keep a cool head about which vertex in $G$ (or in $V_{t_i}$, for that matter) corresponds to which vertex in $V_t$.
  \item You use (or write) a data type for sets.
    For table look-up you can use an associative array (for instance, by making the data structure hashable).
      This is a lot slower and requires much more code, but the result is slightly more readable, in particular in a high-level language with neat syntax.
      A good suggestion is to use the programming language Scala, which combines good abstractions with reasonable running times.
\end{enumerate}

The output of your program is just a number (the size of the maximum independent set).
But you are strongly advised to actually compute the elements of a maximum independent set as well.
(By adding the relevant information to $f_t(U)$ when you traverse the tree decomposition.)
Otherwise your code will be very difficult to debug.

\newpage
\section{Treewidth report}


by Alice Cooper and Bob Marley\sidenote{Complete the report by filling
  in your names and the parts marked $[\ldots]$.
  Remove the sidenotes in your final hand-in.}


\subsection{Results}

The following table gives the indpendence number $\alpha(G)$ (the size of a maximum independent set) for each graph:


\medskip
\begin{tabular}{lccc}
  \toprule
  Instance name & $n$ & $w$ & $\alpha(G)$   \\
  \midrule
  web4 & $5$ & $2$ & $3$ \\ 
  WorldMap & $166$ & $5$ & $78$ \\
  FibonacciTree\_10 & $143$ & $1$ & $72$ \\
  StarGraph\_100 & $101$ & $1$ & $100$ \\
  TutteGraph & $46$ & $5$ & $19$ \\
DorogovtsevGoltsevMendesGraph & $3282$ & $2$ & $2187$ \\
HanoiTowerGraph\_4\_3 & $64$ & $13$ & $16$ \\
TaylorTwographDescendantSRG\_3 & $17$ & $17$ & $6$ \\
CirculantGraph\_20\_5 & $20$ & $2$ & $10$ \\
AhrensSzekeresGeneralizedQuadrangleGraph\_3 & $27$ & $17$ & $6$ \\
DesarguesGraph & $20$ & $6$ & $10$ \\
  FranklinGraph & $12$ & $4$ & $6$ \\
FolkmanGraph & $20$ & $6$ & $10$ \\
  GoldnerHararyGraph & $11$ & $3$ & $6$ \\
FriendshipGraph\_10 & $21$ & $2$ & $10$ \\
  HerschelGraph & $11$ & $3$ & $6$ \\
HoltGraph & $27$ & $9$ & $10$ \\
  Klein7RegularGraph & $24$ & $13$ & $6$ \\
McGeeGraph & $24$ & $7$ & $10$ \\
TaylorTwographSRG\_3 & $28$ & $22$ & $5$ \\
WellsGraph \\
  SierpinskiGasketGraph\_3 & $15$ & $3$ & $6$ \\

  \ldots\\
  \bottomrule
\end{tabular}


\subsection{Our implementation}

We implemented sets as vectors.
The largest $n$ and $w$ for which this implementation worked  in 60 seconds on our
machine was $n=[\cdots]$ and $w=\cdots$ (the graph called $\cdots$), or $n=\cdots$ and $w=\cdots$ (the graph called $\cdots$). Didn't find any but best we could fo under 1 min was with $w=17$ with 25 sec.

\end{document}
 
